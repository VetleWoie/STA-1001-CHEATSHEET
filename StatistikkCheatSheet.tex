\documentclass[8pt,a4paper,twocolumn,twoside]{article}
\usepackage[a4paper,total={20cm, 28.5cm},headsep=0.3cm]{geometry}
\usepackage[utf8]{inputenc}
\usepackage{sectsty}
\usepackage{titlesec}
\usepackage{amsfonts}
\usepackage{enumitem}
\usepackage{amsmath}

\titleformat{\subsection}{\small\bfseries}{\thesection}{1em}{}
\titleformat{\subsubsection}{\footnotesize\bfseries}{\thesection}{1em}{}
\sectionfont{\fontsize{10}{8}\selectfont}

\titlespacing*{\section}{0pt}{0.2cm}{0.1cm}
\titlespacing*{\subsection}{0pt}{0.2cm}{0.1cm}
\titlespacing*{\subsubsection}{0pt}{0.2cm}{0.1cm}

\setlength{\columnseprule}{1pt}


\def\abs#1{\lvert #1 \rvert}
\def\real{\mathbb{R}}
\def\suminfty#1#2{\sum_{n=#1}^\infty #2}
\def\der#1#2#3{\frac{d #1}{d #2}\left( #3 \right)}
\def\parder#1#2#3{\frac{\partial #1}{\partial #2}\left( #3 \right)}

\begin{document}

\section*{Sannsynlighetsfordelinger:}
%
%Binomisk Fordeling:
%
\subsection*{Binomisk Fordeling:}
\begin{enumerate}[topsep=0pt,itemsep=0pt, partopsep=0pt]
    \item n uavhengige delforsøk
    \item Suksess eller ikke
    \item P(A)=p i alle forsøk
\end{enumerate}
\begin{itemize}[topsep=0pt,itemsep=0pt, partopsep=0pt]
    \item $X$ = Antall ganger $A$ intreffer på n forsøk. 
    \item $X\sim binom(n,p)$
    \item $f(x)= \binom{n}{x} \cdot p^x\cdot (1-p)^{n-x},\; x=0,1,2,\dots,n$
    \item $P(X\leq x)= \sum_{k=0}^{x} P(X=k)$
    \item $E(X)=np$
    \item $Var(X)=np(1-p)$
\end{itemize}
%
%Hypergeometrisk
%
\subsection*{Hypergeometrisk:}
\begin{enumerate}[topsep=0pt,itemsep=0pt, partopsep=0pt]
    \item Populasjon med $N$ elementer.
    \item $k$ av disse regnes som "Suksess", $N-k$ som fiasko
    \item Trekker $n$ elementer uten tilbakelegging
\end{enumerate}
\begin{itemize}[topsep=0pt,itemsep=0pt, partopsep=0pt]
    \item X, antallet suksesser.
    \item $f(x)=\frac{\binom{k}{x}\cdot\binom{N-k}{n-x}}{\binom{N}{n}}$
    \item $E(X)=np,\;p=k/N$
    \item $Var(X)=np(1-p)\frac{N-n}{N-1}$
\end{itemize}
%
%Negativ-Binomisk
%
\subsection*{Negativ-Binomisk:}
$X$ er antall forsøk en må gjøre for at en hendelse $A$ skal intreffe $k$ ganger
\begin{itemize}[topsep=0pt,itemsep=0pt, partopsep=0pt]
    \item $f(x) = \binom{x-1}{k-1}\cdot p^x(1-p)^{x-k},\;x=k,k+1,k+2,\dots$
    \item $E(X)=k/p$
    \item $Var(x)=k\cdot\frac{1-p}{p^2}$
\end{itemize}
%
%Geometrisk
%
\subsection*{Geometrisk:}
$X$ er antall forsøk en må gjøre for at hendelsen $A$ intreffer første gang.
\begin{itemize}[topsep=0pt,itemsep=0pt, partopsep=0pt]
    \item $E(X)=1/p$
    \item $Var(X)=\frac{1-p}{p^2}$
\end{itemize}
Geometrisk fordeling er minneløs!
%
%Poisson
%
\subsection*{Poisson:}
Antall forekomster av hendelsen $A$ er Poisson-fordelt hvis:
\begin{enumerate}[topsep=0pt,itemsep=0pt, partopsep=0pt]
    \item Antallet av $A$ i disjunkte tidsintervall er uavhengige
    \item Forventa antall av $A$ er konstant lik $\lambda$(raten) per tidsenhet
    \item Kan ikke få to forekomster samtidig
\end{enumerate}
\begin{itemize}[topsep=0pt,itemsep=0pt, partopsep=0pt]
    \item $X$ = antall forekomster av $A$ i et tidsrom $t$
    \item $f(x) = \frac{(\lambda t)^x\cdot e^{-\lambda t}}{x!},\; x=0,1,2,\dots$
    \item $E(X) = \lambda t$
    \item $Var(X) = \lambda t$
    \item $P(X\leq x)=\sum_{k=0}^x P(X=k)$
\end{itemize}
%
%Normalfordeling
%
\subsection*{Normalfordeling:}
\begin{itemize}[topsep=0pt,itemsep=0pt, partopsep=0pt]
    \item $f(x)=\frac{1}{\sqrt{2\pi}\sigma}e^{-\frac{(x-\mu)^2}{2\sigma^2}}$
    \item $P(a\leq X \leq b) = \int_a^b f(x)dx$
\end{itemize}
\subsubsection*{Standard normalfordeling:}
\begin{itemize}
    \item Alle normalfordelinger kan skrives som Standard normalfordeling
    \item $Z=\frac{X-\mu}{\sigma}$
    \item $F(x)=F(X\leq x) = P\left(\frac{X-\mu}{\sigma}\leq\frac{x-\mu}{\sigma}\right)=P\left( Z\leq \frac{x-\mu}{\sigma} \right)=\Phi\left(\frac{x-\mu}{\sigma}\right)$ 
\end{itemize}
Anta at $X_1,X_2,\dots,X_n$ er uavhengige og normalfordelt. Da er:
$Y=\alpha_1 X_1+\alpha_2 X_2+\dots+\alpha_n X_n$
Være normalfordelt med:
\begin{itemize}[topsep=0pt,itemsep=0pt, partopsep=0pt]
    \item $E(Y)=\sum_{i=1}^n \alpha_i \mu_i$
    \item $Var(Y)=\sum_{i=1}^n \alpha^2_i\sigma^2_i$
\end{itemize}
\end{document}