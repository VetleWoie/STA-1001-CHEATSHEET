\documentclass[8pt,a4paper,twocolumn,twoside]{article}
\usepackage[a4paper,total={20cm, 28.5cm},headsep=0.3cm]{geometry}
\usepackage[utf8]{inputenc}
\usepackage{sectsty}
\usepackage{titlesec}
\usepackage{amsfonts}
\usepackage{enumitem}
\usepackage{amsmath}

\titleformat{\subsection}{\small\bfseries}{\thesection}{1em}{}
\titleformat{\subsubsection}{\footnotesize\bfseries}{\thesection}{1em}{}
\sectionfont{\fontsize{10}{8}\selectfont}

\titlespacing*{\section}{0pt}{0.2cm}{0.1cm}
\titlespacing*{\subsection}{0pt}{0.2cm}{0.1cm}
\titlespacing*{\subsubsection}{0pt}{0.2cm}{0.1cm}

\setlength{\columnseprule}{1pt}


\def\abs#1{\lvert #1 \rvert}
\def\real{\mathbb{R}}
\def\suminfty#1#2{\sum_{n=#1}^\infty #2}
\def\der#1#2#3{\frac{d #1}{d #2}\left( #3 \right)}
\def\parder#1#2#3{\frac{\partial #1}{\partial #2}\left( #3 \right)}
\def\forvar#1#2#3{E(#1)=#2 \quad Var(#1)=#3}

\begin{document}

\section*{Sannsynlighetsfordelinger:}
%
%Binomisk Fordeling:
%
\subsection*{Binomisk Fordeling:}
\begin{enumerate}[topsep=0pt,itemsep=0pt, partopsep=0pt]
    \item n uavhengige delforsøk
    \item Suksess eller ikke
    \item P(A)=p i alle forsøk
\end{enumerate}
\begin{itemize}[topsep=0pt,itemsep=0pt, partopsep=0pt]
    \item $X$ = Antall ganger $A$ intreffer på n forsøk. 
    \item $X\sim binom(n,p)$
    \item $f(x)= \binom{n}{x} \cdot p^x\cdot (1-p)^{n-x},\; x=0,1,2,\dots,n$
    \item $P(X\leq x)= \sum_{k=0}^{x} P(X=k)$
    \item $E(X)=np\quad Var(X)=np(1-p)$
\end{itemize}
%
%Hypergeometrisk
%
\subsection*{Hypergeometrisk:}
\begin{enumerate}[topsep=0pt,itemsep=0pt, partopsep=0pt]
    \item Populasjon med $N$ elementer.
    \item $k$ av disse regnes som "Suksess", $N-k$ som fiasko
    \item Trekker $n$ elementer uten tilbakelegging
\end{enumerate}
\begin{itemize}[topsep=0pt,itemsep=0pt, partopsep=0pt]
    \item X, antallet suksesser.
    \item $f(x)=\frac{\binom{k}{x}\cdot\binom{N-k}{n-x}}{\binom{N}{n}}$
    \item $\forvar{X}{np}{np(1-p)\frac{N-n}{N-1}},\;p=k/N$
\end{itemize}
%
%Negativ-Binomisk
%
\subsection*{Negativ-Binomisk:}
$X$ er antall forsøk en må gjøre for at en hendelse $A$ skal intreffe $k$ ganger
\begin{itemize}[topsep=0pt,itemsep=0pt, partopsep=0pt]
    \item $f(x) = \binom{x-1}{k-1}\cdot p^x(1-p)^{x-k},\;x=k,k+1,k+2,\dots$
    \item $E(X)=k/p\quad Var(x)=k\cdot\frac{1-p}{p^2}$
\end{itemize}
%
%Geometrisk
%
\subsection*{Geometrisk:}
$X$ er antall forsøk en må gjøre for at hendelsen $A$ intreffer første gang.
\begin{itemize}[topsep=0pt,itemsep=0pt, partopsep=0pt]
    \item $E(X)=1/p \quad Var(X)=\frac{1-p}{p^2}$
\end{itemize}
Geometrisk fordeling er minneløs!
%
%Poisson
%
\subsection*{Poisson:}
Antall forekomster av hendelsen $A$ er Poisson-fordelt hvis:
\begin{enumerate}[topsep=0pt,itemsep=0pt, partopsep=0pt]
    \item Antallet av $A$ i disjunkte tidsintervall er uavhengige
    \item Forventa antall av $A$ er konstant lik $\lambda$(raten) per tidsenhet
    \item Kan ikke få to forekomster samtidig
\end{enumerate}
\begin{itemize}[topsep=0pt,itemsep=0pt, partopsep=0pt]
    \item $X$ = antall forekomster av $A$ i et tidsrom $t$
    \item $f(x) = \frac{(\lambda t)^x\cdot e^{-\lambda t}}{x!},\; x=0,1,2,\dots$
    \item $E(X) = \lambda t \quad Var(X) = \lambda t$
    \item $P(X\leq x)=\sum_{k=0}^x P(X=k)$
    \item Ventetida til hendelse k er gammafordelt med $\alpha =k$ og $\beta=1/\lambda$
    \item Ventetida til første hendelse,og mellom etterfølgende hendelser, er eksponensialfordelt
\end{itemize}
%
%Uniform Fordeling:
%
\subsection*{Uniform fordeling:}
En kontinuerlig uniformt fordelt variabel, har samme sannsynlighet for alle
verdier innen et intervall. Generelt har vi tetthetsfunksjonen:
$f(x)=\begin{cases}
    \frac{1}{B-A},\; A\leq x\leq B\\
    0,\; \text{ellers}
\end{cases}$
\begin{itemize}[topsep=0pt,itemsep=0pt, partopsep=0pt]
    \item $E(X)=\frac{A+B}{2} \quad Var(X)=\frac{(A-B)^2}{12}$
\end{itemize}
%
%Gammafordeling
%
\subsection*{Gammafordeling:}
En kontinuerlig variabel $X$ er gammfordelt med parameter $\alpha>0$ og $\beta>0$
dersom tetthetsfunksjonen er gitt ved:
$f(x;\alpha,\beta) = \begin{cases}
    \frac{1}{\beta^\alpha \Gamma(\alpha)} x^{\alpha-1} e^{-\frac{x}{\beta}},\; x>0\\
    0,\;\text{ellers}
\end{cases}$
\begin{itemize}[topsep=0pt,itemsep=0pt, partopsep=0pt]
    \item $E(X)=\alpha\beta \quad Var(X)=\alpha\beta^2$
\end{itemize}
%
%Eksponensialfordeling
%
\subsection*{Eksponensialfordeling:}
\begin{itemize}[topsep=0pt,itemsep=0pt, partopsep=0pt]
    \item $f(x;\beta)=\begin{cases}
        \frac{1}{\beta}e^{-\frac{x}{\beta}},\;x>0\\
        0,\;ellers
    \end{cases}$
    \item $\forvar{X}{\beta}{\beta^2}$
\end{itemize}
Eksponensialfordelinga er minneløs!
%
%Normalfordeling
%
\subsection*{Normalfordeling:}
\begin{itemize}[topsep=0pt,itemsep=0pt, partopsep=0pt]
    \item $f(x)=\frac{1}{\sqrt{2\pi}\sigma}e^{-\frac{(x-\mu)^2}{2\sigma^2}}$
    \item $P(a\leq X \leq b) = \int_a^b f(x)dx$
\end{itemize}
\subsubsection*{Standard normalfordeling:}
\begin{itemize}[topsep=0pt,itemsep=0pt, partopsep=0pt]
    \item Alle normalfordelinger kan skrives som Standard normalfordeling
    \item $Z=\frac{X-\mu}{\sigma}$
    \item $F(x)=F(X\leq x) = P\left(\frac{X-\mu}{\sigma}\leq\frac{x-\mu}{\sigma}\right)=P\left( Z\leq \frac{x-\mu}{\sigma} \right)=\Phi\left(\frac{x-\mu}{\sigma}\right)$ 
\end{itemize}
Anta at $X_1,X_2,\dots,X_n$ er uavhengige og normalfordelt. Da er:
$Y=\alpha_1 X_1+\alpha_2 X_2+\dots+\alpha_n X_n$
Være normalfordelt med:
\begin{itemize}[topsep=0pt,itemsep=0pt, partopsep=0pt]
    \item $E(Y)=\sum_{i=1}^n \alpha_i \mu_i\quad Var(Y)=\sum_{i=1}^n \alpha^2_i\sigma^2_i$
\end{itemize}


%
%Inferens
%
\section*{Inferens:}
%
%QQ-Plot
%
\subsection*{QQ-Plot:}
\begin{itemize}[topsep=0pt,itemsep=0pt, partopsep=0pt]
    \item Plotter observasjoner mot teoretiske("ideelle") observasjoner fra en aktuell fordeling.
    \item Teoretiske observasjoner er gitt ved invers kumulativ fordeling av jevnt spredte Sannsynlighetsfordelinger
          mellom 0 og 1.
    \item Om antatt fordeling stemmer skal plottet gi tilnermet rett linje.
\end{itemize}
%
%Estimering
%
\section*{Estimering:}
%
%Viktige estimatorer
%
\subsection*{Viktige estimatoregenskaper:}
\begin{itemize}[topsep=0pt,itemsep=0pt, partopsep=0pt]
    \item En punktestimator $\Theta$ for en paramaeter $\theta$ er forventningsrett hvis $E(\Theta)=\theta$
    \item Variansen $Var(\Theta)$ burde synke med økende antall observasjoner
    \item Om en har to ulike estimatorer, så er den estimatoren med minst varians den mest effektive estimatoren.
\end{itemize}
%
%Noen vanlige estimatorer
%
\subsection*{Vanlige estimatorer:}
Alle estimatorene vist til her er forventningsrett.
\begin{itemize}[topsep=0pt,itemsep=0pt, partopsep=0pt]
    \item $\mu$: $\overline{X}=\frac{1}{n}\sum_{i=1}^n X_i \quad \forvar{\overline{X}}{\mu}{\frac{\sigma^2}{n}}$
    \item $\sigma^2$: $S^2=\frac{1}{1-n}\sum_{i=1}^n(X_i-\overline{X})^2 \; \forvar{S^2}{\sigma^2}{\frac{2\sigma^4}{n-1}}$
    \item $p$: $\hat{p}=\frac{X}{n} \quad \forvar{\hat{p}}{p}{\frac{p(1-p)}{p}}$ Binomisk
    \item $\mu_1-\mu_2$: $\overline{X_1}-\overline{X_2} \quad Var(\overline{X_1}-\overline{X_2})=\frac{\sigma^2_1}{n_1}-\frac{\sigma^2_2}{n_2}$
    \item $\frac{\sigma^2_1}{\sigma^2_2}$: $\frac{S^2_1}{S^2_2}$
    \item $p_1-p_2$: $\hat{p_1}-\hat{p_2}$, Binomisk
    \item $\mu_D: \overline{D}$
\end{itemize}
%
%Utvalgsfordelinger
%
\subsection*{Utvalgsfordelinger:}
\subsubsection*{$\bold{\underline{\overline{X},Z}}$:}
$\overline{X}\sim N\left(\mu,\frac{\sigma}{\sqrt{n}}\right)\; Z=\frac{\overline{X}-E(\overline{X})}{\sqrt{Var{\overline{X}}}}=\frac{\overline{X}-\mu}{\frac{\sigma}{\sqrt{n}}}\sim N(0,1)$
Selv om populasjonen ikke er normalfordelt gjelder dette når $n\to\infty$.Regner vanligvis tilnærmina for god når $n>30$
\subsubsection*{Sentralgrenseteoremet:}
Når utvalgsstørrelsen $N\to\infty$ så vil $\overline{X}\sim N\left(\mu,\frac{\sigma}{\sqrt{n}}\right)$ for uansett fordeling av $X$. Godkjener dette for $N\geq 30$
\subsubsection*{$\bold{\underline{T}}$}
Hvis ukjent varians:
$T=\frac{\overline{X}-\mu}{S/\sqrt{n}}\sim t_{n-1}$
Dette gjelder tilnærmet andre fordelinger som har klokkeliknende form.
\subsubsection*{$\bold{\underline{S^2}}$:}
Forutsatt normalfordeling:
$\frac{(n-1)S^2}{\sigma^2}=\frac{1}{\sigma^2}\sum_{i=1}^n(X_i-\overline{X})^2\sim\chi_{n-1}^2$
\subsection*{$\bold{\underline{\hat{p}}}$:}
Binomisk forsøk med sannsynlighet $p$, gitt at $n$ er stor nok:
$Z=\frac{\hat{p}-E(\hat{p})}{\sqrt{Var(\hat{p})}}=\frac{\hat{p}-p}{\sqrt{\frac{p(1-p)}{n}}}\sim N(0,1)$
\subsection*{$\bold{\underline{\hat{X_1}-\hat{X_2}}}$:}
Kjent varians:
\begin{itemize}
    \item $\hat{X_1}-\hat{X_2} \sim N\left(\mu_1-\mu_2,\sqrt{\frac{\sigma_1^2}{n_1}+\frac{\sigma_2^2}{n_2}}\right)$
    \item $Z=\frac{(\overline{X_1}-\overline{X_2})-(\mu_1-\mu_2)}{\sqrt{\frac{\sigma^2_1}{n_1}+\frac{\sigma^2_2}{n_2}}}\sim N(0,1)$
\end{itemize}
Ukjent varians:
\begin{itemize}[topsep=0pt,itemsep=0pt, partopsep=0pt]
    \item $\sigma^2_1=\sigma^2_2$: $T=\frac{(\overline{X_1}-\overline{X_2}-(\mu_1-\mu_2)}{S_p\sqrt{\frac{1}{n_1}+\frac{1}{n_2}}}\sim t_{n_1+n_2-2}$
    \item $\sigma^2_1\neq\sigma^2_2$: $T'=\frac{(\overline{X_1}-\overline{X_2})-(\mu_1-\mu_2)}{\sqrt{\frac{S^2_1}{n_1}+\frac{S^2_2}{n_2}}}\sim t_v,\, v=\frac{(S^2_1/n_1+s^2_2/n_2)^2}{\frac{(S^2_/n_1)^2}{n_1-1}+\frac{(S^2_2/n_2)^2}{n_2-1}}$
\end{itemize}
\subsection*{$\bold{\underline{F}}$}
Fra to uavhengige NF utvalg:
$F=\frac{S^2_1\sigma^2_2}{S^2_2\sigma^2_1}\sim F_{n_1-1,n_2-1}$
\subsection*{$\bold{\underline{\hat{p_1}-\hat{p_2}}}$}
$Z=\frac{(\hat{p_1}-\hat{p_2})-(p_1-p_2)}{\sqrt{\frac{p_1(1-p_1)}{n_1}+\frac{p_2(1-p_2)}{n_2}}}\sim N(0,1)$
\subsection*{$\bold{\underline{\overline{D}}}$}
Gitt normalfordeling: $T=\frac{\overline{D}-\mu_D}{S_D/\sqrt{n}}\sim t_{n-1}$\\
Fra utvalgsfordelinger kan en utlede testobservatorer og konfidensintervaller!
%
%Oppskrift; Utlede konfidensintervall.
%
\section*{Utlede Konfidensintervall:}
Anta at vi har $X_1,X_2,\dots,X_n$ stokastiske variabler, hvor sannsynlighetsfordelingen til disse inneholder
en ukjent parameter $\theta$. Anta også at vi har observasjoner $x_1,x_2,\dots,x_n$. Har lyst å bruke disse 
for å finne et $100(1-\alpha)\%$ konfidensintervall:
\begin{enumerate}[topsep=0pt,itemsep=0pt, partopsep=0pt]
    \item Bestem en stokastiske variabel $Z = h(X_1,X_2,\dots,X_n,\theta)$ som følger en kjent fordeling. Altså finn
     utvalgsfordelingen for paramaeteren $\theta$
    \item Finn kvantilene $Z_{\frac{\alpha}{2}}$ og $Z_{1-\frac{\alpha}{2}}$. Da har en at: $P(Z_{1-\frac{\alpha}{2}}\leq h(X_1,X_2,\dots,X_n,\theta)\leq Z_{\frac{\alpha}{2}})$
    \item Da er løsningen på ulikhetene $Z_{1-\frac{\alpha}{2}} \leq h(X_1,X_2,\dots,X_n,\theta)$ og $Z_{1-\frac{\alpha}{2}}\geq h(X_1,X_2,\dots,X_n,\theta)$ konfidensintervallet.
\end{enumerate}
%
%Prediksjonsintervall
%
\section*{Prediksjonsintervall:}
\subsection*{$\bold{\mu,\sigma}$ kjent:}
$P\left(-z_{\frac{\alpha}{2}}\leq\frac{X_0-\mu}{\sigma}\leq z_{\frac{\alpha}{2}}\right) = 1-\alpha$
\subsection*{$\bold{\mu}$ kjent og $\bold{\sigma}$ ukjent:}
$P\left(-z_{\frac{\alpha}{2}}\leq\frac{X_0-\overline{X}}{\sigma\sqrt{1+\frac{1}{n}}}\leq z_{\frac{\alpha}{2}}\right) = 1-\alpha$
\subsection*{$\bold{\mu,\sigma}$ ukjent:}
$P\left(-t_{\frac{\alpha}{2}}\leq\frac{X_0-\overline{X}}{S\sqrt{1+\frac{1}{n}}}\leq t_{\frac{\alpha}{2}}\right) = 1-\alpha$

%
%Hypotesetesting
%
\section*{Hypotesetesting:}
Velger testobservator med kjent fordeling(Velger utvalgsfordelingen til parameteren) når nullhypotesen er sann.
Dersom utregna testobservator gir en verdi som er veldig usansynlig hvis nullhypotesen er sann forkastes nullhypotesen.
\subsection*{Forkastningsområde:}
Forkastnings område velges slik at det skal være en sannsynlighet $\alpha$ for å få en så ekstrem verdi, dersom $H_0$ er sant.
Kritisk verdi blir da $z_\alpha$, og vi forkaster $H_0$ om $Z>z_\alpha$
%
%Enkel lineær regresjon:
%
\section*{Enkel lineær regresjon:}
\subsection*{Regresjonsmodell:}
\begin{itemize}[topsep=0pt,itemsep=0pt, partopsep=0pt]
    \item $Y_i)\beta_0+\beta_1x_i+\epsilon_i,\;i=1,\dots,n$
    \item Forutsatt: $E(\epsilon_i) = 0,\;Var(\epsilon_i)=\sigma^2$
    \item $E(Y_i)=\mu_{Y|x_i}=\beta_0+\beta_1x_i,\;Var(Y_i)=\sigma^2_{Y|x_i}=\sigma^2$
\end{itemize}
\subsection*{Minste kvadraters metode:}
\begin{itemize}[topsep=0pt,itemsep=0pt, partopsep=0pt]
    \item Brukes for å finne $\beta_0,\beta_1$ fra data
    \item Vil minimere $SSE=\sum_{i=1}^n(y_i-\hat{y_i})^2=\sum_{i=1}^n(y_i-b_0-b_1)$
    \item Minste verdier av $b_0,b_1$ kan finnes ved partiellderivasjon.
    \item $b_1=\frac{\sum_{i=1}^n(x_i-\overline{x})(Y_i-\overline{Y})}{\sum_{i=1}^n(x_i-\overline{x})^2},\; b_0=\overline{Y}-B_1\overline{x}$
    \item $b_0, b_1$ forventningsrette estimatorer for $\beta_0,\beta_1$
    \item $S^2=\frac{SSE}{n-2}$
\end{itemize}
\subsubsection*{Inferens av parametere:}
\begin{itemize}[topsep=0pt,itemsep=0pt, partopsep=0pt]
    \item $T=\frac{b_1-E(b_1)}{\hat{SE(b_1)}}=\frac{b_1-\beta_1}{\frac{S}{\sqrt{\sum_{i=1}^n(x_i-\overline{x})^2}}}$, t-fordelt,$n-2$ frihetsgrader
    \item $T=\frac{b_0-E(b_0)}{\hat{SE(b_0)}}=\frac{b_0-\beta_0}{S\sqrt{\frac{\sum_{i=1}^nx^2_i}{n\sum_{i=1}^n(x_i-\overline{x})^2}}}$,t-fordelt, $n-2$ frihetsgrader
\end{itemize}
\subsubsection*{Inferens av $\bold{\sigma^2}$ og $\bold{\mu_{Y|x_0}}$:}
\begin{itemize}[topsep=0pt,itemsep=0pt, partopsep=0pt]
    \item $\sigma^2$: $V=\frac{(n-2)S^2}{\sigma^2}$, $\chi^2$-fordelt, $n-2$ frihetsgrader
    \item $\mu_{Y|x_0}$: $T=\frac{\hat{Y_0}-\mu_{Y|x_0}}{S\sqrt{\frac{1}{n}+\frac{(x_0-\overline{x})^2}{\sum_{i=1}^n(x_i-\overline{x})^2}}}$, t-fordelt, $n-2$ frihetsgrader
\end{itemize}

%
%Generell Sannsynlighet og statistikk
%
\section*{Generell sannsynlighet og statistikkregler:}
\begin{itemize}[topsep=0pt,itemsep=0pt, partopsep=0pt]
    \item Addisjonsregel $P(A\cup B)=P(A)+P(B)-P(A\cap B)$
    \item Betinget sannsynlighet $P(A|B)=\frac{P(A\cap B)}{P(B)}$
    \item Multiplikasjonsregelen:$P(A\cap B)=P(B)P(A|B)=P(A)P(B|A)$
    \item Kumulativ fordelingsfunksjon $F(X)=P(X\leq x)=\begin{cases}
        \sum_{t\leq x} P(X=t),\;\text{diskret}\\
        \int_{-\infty}^x f(t)dt,\;\text{Kont.}
    \end{cases}$
    \item Forventningsverdi:$E(X)=\begin{cases}
        \sum_x xf(x),\;\text{Diskret}\\
        \int_{-\infty}^\infty,\;\text{Kont.}
    \end{cases}$
    \item Varians:\\$Var(X)=\begin{cases}
        \sum_x(x-\mu)^2f(x)=E(X^2)-E(X)^2\;\text{Diskret}\\
        \int_{-\infty}^\infty (x-\mu)^2f(x)dx = \int_{-\infty}^{\infty}x^2f(x)dx\cdot\mu^2,\;\text{Kont.}
    \end{cases}$
\end{itemize}
%
%Simultanfordeling for to variabler
%
\section*{Simultanfordeling for to variabler:}
\begin{itemize}[topsep=0pt,itemsep=0pt, partopsep=0pt]
    \item $P((x,y)\in A)=\begin{cases}
        \int\int_A f(x,y)dxdy\;\text{Kont.}\\
        \sum\sum_A F(x,y)\;\text{Diskret}
    \end{cases}$
    \item $g(x)=\int_{-\infty}^\infty f(x,y)dy\;\text{eller}\;\sum_yf(x,y)$ Samme for $h(y)$
    \item $f(y|x)=\frac{f(x,y)}{g(x)}$
    \item $\sigma_{XY}=Cov(X,y)=E[(X-\mu_X)(Y-\mu_Y)]=\begin{cases}
        \sum_x\sum_y(x-\mu_X)(y-\mu_Y)f(x,y),\;\text{Diskret}\\
        \int_{-\infty}^\infty\int_{-\infty}^\infty (x-\mu_X)(y-\mu_Y)f(x,y)dxdy\;\text{Kont.}
    \end{cases}$
    \item $\rho_{XY}=Cor(X,Y)=\frac{Cov(X,Y)}{\sqrt{Var(X)Var(Y)}}$
    \item $X,Y$ uavhengig $\Rightarrow Cov(x,y)=0$
\end{itemize}


\end{document}
